\section{Johnson - Simulated Annealing}

Johnson et al.'s three part series on Simulated Annealing published in 1991 is perhaps the seminal work on the subject. The second part of the series covers the well studied yet never satisfactorily solved \emph{Graph Coloring Problem}. You didn't see that one coming did you?\\

The paper begins by introducing introducing the context, including what the GCP is, and presenting a general simulated annealing algorithm.

So what is simulated annealing? Simulated annealing is an adjustment to local optimisation that has the ability to escape local optima (to a point). By using a temperature parameter that reduces as the algorithm progresses to govern when an uphill move can be made, simulated annealing is able to move around the solution space easily at early iterations but becomes locked in at later iterations.

In order to approach the GCP from an optimisation stand point three components of the optimisation scheme have to be established: a neighbourhood graph that describes the solution space, a way to move through the solution space, and an initial solution.

Three types of neighbourhoods and movement strategies are proposed, leading to three different SA algorithms for solving the GCP. All algorithms use some sort of randomised initial solution suitable for their particular implementation.

\subsection{Penalty Function approach}

\begin{itemize}
	\item A solution will be \emph{any} partition of $V$ into nonempty disjoint sets $C_1,C_2,\ldots,C_k$, whether the $C_i$ are legal color classes or not.
	\item Two solutions are neighbours if one can be transformed to the other by moving a vertex from one color class to another.
	\item To generate a random neighbour by randomly picking a nonempty color class $C_{OLD}$, a vertex $v \in C_{OLD}$, and then an integer $i,\,1 \leq i \leq k+1$. The neighbour is obtained by moving $v$ to $C_i$. If $i = k+1$ then $v$ is moved to a new, previously empty class.
	\item Note that this biases the choice of $v$ towards vertices in smaller color classes.
	\end{itemize}


\begin{itemize}
	\item Adopts the general philosophy of RLF, which constructs its colorings with the aid of a subroutine for generating large independent sets.
	\item Cost function has two components, the first favours large color classes, the second favours independent sets.
	\item Let $\Pi = (C_1,\ldots,C_k)$ be a solution and $E_i, \, 1 \leq i \leq k$ be the set of edges from $E$ both of whose endpoints are in $C_i$, i.e., the set of \emph{bad} edges in $C_i$.
	\item Cost Function
		\begin{equation*}
		\centering
		cost(\Pi) = - \displaystyle \sum_{i=1}^{k} |C_i|^2 + \sum_{i=1}^{k} 2|C_i| \dot |E_i|
		\end{equation*}
	\end{itemize}

\subsection{Kempe Chain approach}

\begin{itemize}
\item Solutions are now restricted to be partitions $C_1,C_2,\ldots,C_k$ that are legal colorings.
\item Cost function simplifies to
\begin{equation*}
cost(\Pi) = - \displaystyle \sum_{i=1}^{k} |C_i|^2.
\end{equation*}
\item Moves through the solution space are based on \emph{Kempe Chains}
\end{itemize}

\begin{itemize}
	\item Suppose that $C$ and $D$ are disjoint independent sets in a graph $G$.
	\item A Kempe chain for $C$ and $D$ is any connected component in the subgraph of $G$ induced by $C \cup D$.
	\item Let $X \Delta Y$ denote the symmetric difference $(X-Y)\cup(Y-X)$.
	\item Let $H$ be a Kempe chain of $C$ and $D$, then $C \Delta H$ and $D \Delta H$ are themselves disjoint independent sets whose union is $C \cup D$.
\end{itemize}

\begin{itemize}
	\item Randomly choose a nonempty color class $C$ and a vertex $v \in C$.
	\item Randomly choose a nonempty color class $D$ other than $C$
	\item Let $H$ be the \emph{Kempe chain} for $C$ and $D$ that contains $v$.
	\item Repeat until one obtains $C , D , v$ and $H$ s.t. $H \neq C \cup D$
	\item The next partition is obtained by replacing $C$ by $C \Delta H$ and $D$ by $D \Delta H$ 
	\end{itemize}

\subsection{Fixed-K approach}
	\begin{itemize}
	\item Takes a different approach to the previous methods. As the name suggests, this algorithm runs for some fixed $k$.
	\item Solutions are any partitioning of $V$ into $k$ color classes.
	\item Attempts to minimise the number of \emph{bad} edges.
	\item A partition $\Pi_2$ is a neighbour of a partition $\Pi_1$ if the two partitions differ only as to the position of a single vertex $v$ and $v$ is an endpoint of a bad edge in $\Pi_1$.
	\item A legal coloring has no neighbours since if a legal coloring is found the algorithm stops.
	\end{itemize}

\subsection{Testing}

In order to form a comparison to the current methodologies of the time, three Successive Augmentation algorithms are also run: DSATUR, RLF and XRLF (an extension of the former).
All algorithms are run on random graphs of various sizes and edge probabilities, some ``cooked'' graphs that have chromatic numbers approximately half the equivalent random graph, and some geometric graphs that have certain properties as well as their complements.

\subsection{Author's Conclusions}

\begin{itemize}
	\item Experimental results did not allow the authors to identify a best graph coloring heuristic. Rather, they feel that their results reinforce the notion that there is no best heuristic.
	\item In general Kempe chain annealing starts to take over fixed-K annealing as density increases. This makes sense given their neighbourhood structure.
	\item Penalty function annealing, Kempe chain annealing and RLF are biased towards favouring unbalanced colorings, i.e. better a large and a small sized class than two equally sized classes. Fixed-K on the other hand is neutral, potentially giving it an advantage on graphs whose good colorings are balanced.
	\end{itemize}

\subsection{Wrap-up}

Johnson et al. 1991 is a highly regarded and often referenced mainstay in the field of optimisation for good reason. The paper is well written, describes the methods in good detail and reports on extensive experimentation. 
On a personal note, I was very disappointed that the manuscript by Shapiro and Morgenstern that Johnson et al. cites for their Kempe Chain implementation is not available online and in fact only exists as a physical copy in three locaitons in the world: Centrum Wiskunde en Informatica - Amsterdam, NL 1098 XG Netherlands; University of New Mexico-Main Campus - Albuquerque, NM 87131 United States; and Stanford University Libraries - Stanford, CA 94305 United States.



