\section{Flower Pollination}
\subsection{Implementation and Problems}

Ok so I have to say that I think that this paper\cite{7175923} SUCKS. I was drawn to it initially because it was a new (2015) optimisation algorithm and we wanted something to compare to the gravitational swarm intelligence\cite{bib:GravSwarm}, plus it was flowers which is cute.
Anyway, the first warning sign should have been that it's from Algeria, but that would have been racist. Second warning sign: THEY WORD FOR WORD COPIED from Yang's paper. Like, full blown plagiarism. Damn, how does that pass peer review??? Of course, I hadn't actually read Yang's paper at the start so I didn't know that.. Third warning sign: The algorithm doesn't quite match up with the pollination properties it supposedly takes inspiration from. Ok, not just these guys fault, Yang does the same thing.
So I tried to implement it. And I failed. Some parts of the pseudocode I just didn't know what they were talking about, and when I managed to work around that to hobble something together that ran, it just didn't do anything. So I hacked and hacked and made something that started to work! But it was still rubbish.

The original algorithm is as follows


\begin{algorithm}[H]
    \caption{FPA for GCP}
    \label{FPA1}
    \begin{algorithmic}[1] % The number tells where the line numbering should start
        \Procedure{solve}{Graph $g$, $iterationLimit$, $numFlowers$} \Comment{$g$ is predefined}			
			\State Initialize a population of $N$ flowers/pollen gametes with random solutions
			\State Find the best solution $g_*$ in the initial population
			\State Define a switch probability $p \in [0,1]$
			\State $t=0$
			\While{$t < iterationLimit$}   
				\For{$i=1:N$}
					\If{$rand < p$} \Comment{$rand$ is a function that returns a single uniformly distributed random number between 0 and 1}
						\State Draw a (d-dimensional) step vector $L$ which obeys a L\`evy distribution.
						\State Global pollination via $x_i^{t+1} = x_i^{t} + L(x_i^{t} - g_*)$
					\Else
						\State Draw $U$ from a uniform distribution $[0,1]$
						\State Do local pollination via $x_i^{t+1} = x_i^{t} + \epsilon(x_j^{t} - x_k^{t})$
					\EndIf
					\State //Discretisation and correction step
					\State Apply round function to obtain integer values
					\State Correction Step() \Comment{To ensure that no invalid colors are used.}
					\If{all $k$ colors are assigned}
						\State Swap the most conflicting node with a random vertex of the lowest used color
						\State Evaluate new solutions $x^{t+1}$ using global pollination
						\If{$f(x^{t+1})<f(x^{t})$}\Comment{If new solutions are better}
							\State $x^t = x^{t+1}$
						\EndIf
					\Else continue
					\EndIf
				\EndFor
				\State Update the current best solution $g_*$
					\If{$f^* == 0$}
					\Comment{Legal coloring found}
						\State $k = k-1$
						\State \textbf{Break}
					\EndIf
			\State $t = t+1$
			\EndWhile
			\State Output the best solution found
		\EndProcedure
    \end{algorithmic}
\end{algorithm}


The Correction Step() is also described, but it's so simple I'm not sure why they bothered. If a color class is assigned that is greater than $k$ instead set it to $k$, if a color class is assigned that is less than zero instead set it to zero.
Note that the \textbf{Break} when a legal coloring for a given $k$  is found should probably be a \textbf{Return} since it is implied (poorly) that this solve procedure exists within a larger framework for descending $k$.


\subsection{The Hacking}

Since the paper does such a poor job of describing the process, I was unable to implement the algorithm in such a way that it was capable of finding even a $k = \Delta$ coloring in an arbitrary \emph{reasonable} time (couple of hours). Frustrated, I made some rather drastic changes without much justification. I will now describe the final algorithm that \emph{sometimes} beats the genetic algorithm (see results).

First let's look at the algorithm

\begin{algorithm}[H]
    \caption{modified FPA for GCP}
    \label{FPA2}
    \begin{algorithmic}[1] % The number tells where the line numbering should start
        \Procedure{solve}{Graph $g$, $iterationLimit$, $numFlowers$} \Comment{$g$ is predefined}			
			\State Initialize the flowerbed
			\State Update the current best solution $g_*$
			\State Define a switch probability $p \in [0,1]$
			\State $t=0$
			\While{$t < iterationLimit$}   
				\For{$flowerNum = 0; flowerNum < numColours; flowerNum++$}
					\State fixColors(flowerNum)
					\If{$rand < p$} \Comment{$rand$ is a function that returns a single uniformly distributed random number between 0 and 1}
						\State Draw a (d-dimensional) step vector $L$ which obeys a L\`evy distribution.
						\State Global pollination via $x_i^{t+1} = x_i^{t} + L(x_i^{t} - g_*)$
					\Else
						\State Draw $U$ from a uniform distribution $[0,1]$
						\State Do local pollination via $x_i^{t+1} = x_i^{t} + \epsilon(x_j^{t} - x_k^{t})$
					\EndIf
					\State fixColors(flowerNum)
					\State Apply round function to obtain integer values
					\State Correction Step \Comment{To ensure that no invalid colors are used.}
					\For{$col = 0 : \alpha$} \Comment{$\alpha$ could be anything, currently set to $k$}
						\State swap(flowerNum)
					\EndFor
				\EndFor
				\State Update the current best solution $g_*$
				\If{$f^* == 0$}
					\State $k = k-1$
					\State \textbf{Return}
				\EndIf

				doNaturalSelection()
				\State $t = t+1$
			\EndWhile
			\State Output the best solution found
		\EndProcedure
    \end{algorithmic}
\end{algorithm}

Ok so as mentioned before, we call the set of flowers the \emph{flowerbed}. The main differences are as follows:

fixColors
In the original algorithm, each flower is checked to see if every $k$ color is assigned. Only if that returns true is the swap strategy and global pollination perturbation performed. I found this to be odd, but rather than remove it I went the other direction and wrote a function that ensured that all colors are present. Simply put, fixColor searches the flower for each color in turn and if a color is not present it assigns that color to a random vertex.

\begin{algorithm}[H]
    \caption{fixCol}
    \label{FPA2.1}
    \begin{algorithmic}[1] % The number tells where the line numbering should start
        \Procedure{fixCol}{$flowerNum$}
			\State how to put?
		\EndProcedure
    \end{algorithmic}
\end{algorithm}

fixColors is run before and after pollination to ensure that every flower adheres to the current $k$.

Rather than performing swap once, under the suspicion that swap is the main mover in the algorithm, I perform swap an arbitrary $\alpha$ number of times. For the purpose of experimentation, of which we report the results on later,  $\alpha = k$ without justification. Swap is directly reducing the cost and so moves the solution vector very quickly through the solution space towards a, presumably, local optima.

doNaturalSelection()
Despite regularly referring to, and comparing against, the genetic algorithm, the authors chose not to implement natural selection or wisdom of crowds. I didn't have the motivation to work out how wisdom of crowds would work for the flowerbed, however implementing natural selection was incredibly straightforward.
The cost of each flower in the flowerbed is assessed and the median value calculate. Every flower with a cost higher than the median value is replaced with a randomly generated flower.
Doing this is a response to the tending towards homogeneity that we've remarked upon before. Every iteration the available genetic material gets a massive shake-up, providing potentially new color partitions that might lead to a legal coloring.

\begin{algorithm}[H]
    \caption{doNaturalSelection}
    \label{FPA2.2}
    \begin{algorithmic}[1] % The number tells where the line numbering should start
        \Procedure{doNaturalSelection}{}
			\State how to put?
		\EndProcedure
    \end{algorithmic}
\end{algorithm}


That's pretty much it for flower pollination.









