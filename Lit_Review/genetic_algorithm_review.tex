\section{Genetic Algorithm}
\subsection{Introduction}
The genetic algorithm presented in this review is taken from this paper \cite{bib:GeneticAlg}. The paper presents a hybrid technique that applies a genetic algorithm followed by wisdom of crowds voting. The algorithm uses a variety of parent selection, child production and mutation steps. The methods of parent selection, child production and mutation vary according to the state of the fitness of the overall system. This results in an algorithm that is resistant to capture by local optima and allows for the solution space to 'jitter' around before falling upon a global solution. 

\subsection{Algorithm Discussion}
The algorithm is presented in further detail in the Implementations section. The following discussion focuses on algorithm design and fine tuning for the GCP. The algorithm makes careful note that local optima are a concern to this algorithm, and the danger posed to any GCP solving heuristic algorithm is falling into a local optima and not being able to recover. The algorithm uses several approaches to avoid local optima and to that end, a number of different factors considered. 

The crossover function, the method that produces a new chromosome from two previously existing chromosomes is the result of a tournament between parents, where 4 chromosomes are chosen at random and the best for the first pair becomes the first parent, and the best of the second pair becomes the second parent. The crossover function takes the chromosomes of the parents and cuts them both at the same point. The new chromosome is the first part of the first parent's chromosome with the second part of the second parents chromosome appended to the end. This results in a new chromosome of the same length as the original parents chromosomes, and a potentially vastly different 'bad edge' count. This new chromosome can then be subjected to mutation (at some rate, given to be 0.7 here). This mutation method examines each vertex in the chromosome and tries to improve it if it is part of a 'bad edge'. 

This process is shown to be very efficient at improving the solution space. It falls prone to the problem of local optima though, so an alternative parent selection and mutation method are employed. These new methods simply take the best solution and mutate it. The mutation allows the solution to become worse, instead of simply trying to improve the solution, each bad edge is allowed to become any colour. This results in the best solution having the possibility of improving, but most likely actually becoming worse. This is not a bad thing though, as the solution space is shown to improve over time. 

If no valid solution to the colouring is found after a set number of generations (given to be 20,000 in this paper), then a wisdom of crowds approach is implemented, where the best portion (given to be half) of the chromosomes vote to create a new chromosome. This aggregated chromosome is checked to see if it contains a solution to the graph, if not, it becomes the first chromosome is the next round of generations, and all other chromosomes are discarded. 

\subsection{Observations}
This algorithm can become very stagnated when the best solution is very close to an optimal solution. To improve this part of the algorithm, the algorithm implementation was changed to instead choose the best parent and a random chromosome on some chance. 