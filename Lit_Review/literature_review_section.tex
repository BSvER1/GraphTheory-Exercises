\chapter{Literature Review}

\section{Introduction}

The graph coloring problem has widespread applications in areas such as scheduling and timetabling.

Bondy and Murty describe graph coloring as follows:
\begin{quotation}
A $k-vertex colouring$ of $G$ is an assignment of $k$ colours, $1,2,\ldots,k$, to the vertices of $G$; the colouring is \emph{proper} if no two distinct adjacent vertices have the same colour. Thus a proper $k$-vertex colouring of a loopless graph $G$ is a partition $(V_1,V_2,\ldots,V_k)$ of $V$ into $k$ (possibly empty) independent sets. $G$ is $k-vertex-colourable$ if $G$ has a proper $k$-vertex colouring. It will be convenient to refer to a `proper $k$-vertex colouring as a $k-colouring$'.
The \emph{chromatic number}, $\chi(G)$, of $G$ is the minimum $k$ for which $G$ is $k$-colourable; if $\chi(G)=k$, $G$ is said to be $k-chromatic$
\end{quotation}

The Graph Coloring Problem (GCP) is the task of trying to find the value for $\chi(G)$ for some given $G$, however, this is an NP-Hard problem of at least $O(N)$ complexity and so finding a $k$ that is guaranteed to be $\chi(G)$ is often ($N>$ single digits) impossible. As such we hope to be able to develop heuristic algorithms that are able to find near-optimal colorings ($k \rightarrow \chi(G)$) quickly (seconds/minutes/hours depending on the size and complexity of the graph).

This work is laid out as follows:
In chapter 1 we will introduce four different methods for solving the GCP that have been proposed by other authors: Simulated Annealing, Gravitational Swarm Intelligence, Genetic Algorithm and Flower Pollination Algorithm. The papers that describe these algorithms, as well as the algorithms themselves, vary in quality and we will be commenting on particularly bad or confusing aspects as they crop up.
We attempted to implement the latter three methods in Java going off of just what was provided in the papers, to mixed success. A write up of these attempts, as well as some pseudocode, is presented in chapter 2.
Finally, chapter 3 has the results of applying the three algorithms to some of the DIMACS benchmark graphs.


\section{Johnson - Simulated Annealing}

Johnson et al.'s three aprt series on Simulated Annealing published in 1991 is perhaps the seminal work on the subject. The second part of the series covers the well studied yet never satisfactorily solved \emph{Graph Coloring Problem}. You didn't see that one coming did you?\\

The paper begins by introducing introducing the context, including what the GCP is, and presenting a general simulated annealing algorithm.

So what is simulated annealing? Simulated annealing is an adjustment to local optimisation that has the ability to escape local optima (to a point). By using a temperature parameter that reduces as the algorithm progresses to govern when an uphill move can be made, simulated annealing is able to move around the solution space easily at early iterations but becomes locked in at later iterations.

In order to approach the GCP from an optimisation stand point three components of the optimisation scheme have to be established: a neighbourhood graph that describes the solution space, a way to move through the solution space, and an initial solution.

Three types of neighbourhoods and movement strategies are proposed, leading to three different SA algorithms for solving the GCP. All algorithms use some sort of randomised initial solution suitable for their particular implementation.

\subsection{Penalty Function approach}
see presentation

\subsection{Kempe Chain approach}
see presentation

\subsection{Fixed-K approach}
see presentation

\subsection{testing}

In order to form a comparison to the current methodologies of the time, three Successive Augmentation algorithms are also run: DSATUR, RLF and XRLF (an extension of the former).
All algorithms are run on random graphs of various sizes and edge probabilities, some ``cooked'' graphs that have chromatic numbers approximately half the equivalent random graph, and some geometric graphs that have certain properties as well as their complements.

\subsection{conclusion}

There is no clear winner. Kempe chains and fixed-k trade off depending on the graph density, the successive augmentation algorithms win at odd times.

\subsection{critique}
I wish I'd been able to implement Kemp chain annealing, it's such a cool idea and was able to find colorings on huge graphs that none of the others could even come close to.\\
All of these are sloooooooooow - more modern optimisation techniques, although still kinda slow, absolutely destroy these algorithms. SA in general is actually pretty bad, even the modern and highly confusing quantum tunnelling SA algorithm gets pantsed by the gravitational swarm. And yet oddly, none of them can hope to compete with Wendy's random buckets... such a strange state of affairs.
\section{Gravitational Swarm Intelligence}
\paragraph*{} The natural inspiration of this algorithm does not come from living beings, such as ants or bees, but from the basic physical law of gravitational attraction between objects. We construct a world that agents navigate through, attracted by the gravitational pull of specific objects, the color goals, such that they may suffer specific repulsion forces, activated by the friend-or-foe nature of the relation between agents induced by the adjacency relation in the underlying graph.
\paragraph*{} \textbf{Initial definition}: Let $G = (V,E)$ be a graph defined on a set of nodes $V = \lbrace v_1,\ldots,v_N \rbrace$ and edges $E \subseteq V \times V$. We define a group of GS-GC agents $B = \lbrace b_1, b_2, \ldots , b_N \rbrace$ each corresponding to a graph node. Each agent navigates inside a square planar toric world according to a speed vector $\overrightarrow{v_i}$. At any moment in time we know the position attribute of each agent $p_i(t) = (x_i,y_i)$ where $x_i$ and $y_i$ are the Cartesian coordinates in the space. When $t = 0$ we have the initial position of the agents $p_i(0) = (x_0, y_0)$. Suppose that we want to color the graph with $K$ colors, denoting $C = \lbrace 1, 2, \ldots , K \rbrace$ the set of colors, where K must not be lower than the chromatic number of the graph for the GS-GC to converge. We assign to these colors, $K$ fixed points in space, the color goals $CG = \lbrace g_1 , \ldots , g_K \rbrace$, endowed with a gravitational attraction resulting in a  velocity component $\overrightarrow{v_{gc}}$ affecting the agents. The attraction force decreases with the distance, but affects all the agents in the space.
\paragraph*{} The problem collapses into the minimisation of a cost function:
\[ min | \lbrace b_i \textbf{ s.t. } b_i \in \lbrace g_1, \ldots g_N \rbrace \rbrace | \]
We denote the set of agents whose position is in the region of the space near enough to a color neighbourhood of the color as:

\begin{equation}
\mathcal{N} \left( g_k \right) = \lbrace b_i \textbf{ s.t. } \Vert p_i - g_k \Vert < nearenough \rbrace
\end{equation}

We denote the fact that the node has been assigned to the corresponding color assigning value to a the agent color attribute

\begin{equation}
b_i \in \mathcal{N}\left( g_k \right) \Rightarrow c_i = k
\end{equation}

The initial value of the agent color attribute $c_i$ is zero or null. Inside the spatial neighbourhood of a color goal there is no further gravitational attraction. However, there may be a repulsion force between agents that are connected with an edge in the graph G. This repulsion is only effective for agents inside the same color goal neighbourhood. To model this effect, we define function repulsion which has value 1 if a pair of GS-GC agents have an edge between them, and 0 otherwise. The repulsive forces experimented by agent $b_i$ from the agents in the color goal $g_k$ are computed as follows:

\begin{equation}
R \left( b_i, g_k \right) = \sum\limits_{\mathcal{N}\left( g_k \right)} repulsion \left( b_i, b_j \right)
\end{equation}

The cost function defined on the global system spatial configuration is:

\begin{equation}
f \left( B, CG \right) = | \lbrace b_i \textbf{ s.t. } c_i \in C \& R \left(b_i, g_{c_i} \right) = 0 \rbrace |
\end{equation}

This cost function is the number of graph nodes which have a color assigned and no conflict inside the color goal. The agents outside the neighbourhood of any color goal can't be evaluated, so it can be a part of the solution of the problem. The dimension of the world and the definition of the $nearenough$ threshold allows controlling the speed of convergence of the algorithm. If the world is big and the $nearenough$ variable is small then the algorithm converges slowly but monotonically to the solution, if the world is small and the $nearenough$ variable is big the algorithm is faster but convergence is jumpy because the algorithm falls in local minima and needs transitory energy increases to escape them. The reason of this behaviour is that the world is not normalized and the magnitude of the velocity vector can be larger than the radius of the color goal's spatial influence and this means an agent could potentially cross a goal without being captured by it.
\paragraph*{} Each color goal has an attraction well spanning the entire space, therefore the gravitational analogy. But in our approach the magnitude of the attraction drops proportionally with the Euclidean distance $d$ between the goal and the GS-GC agent, but it never disappears. If $\Vert d \Vert < nearenough$ then we make $d = 0$, and the agent's velocity becomes 0, stopping it.

We now present a more formal definition of the algorithm.
\paragraph*{} \textbf{Definition}: A Gravitational Swarm (GS) is a collection of particles $P = \lbrace p_1, \ldots , p_L \rbrace$ moving in an space $S$ subjected to attraction and repulsion forces. Attraction correspond to long range gravitational interactions. Repulsions correspond to short range electrical interactions. Particle attributes are: spatial localization $s_i \in S$, mass $m_i \in \mathbb{R}$, charge $\mu_i \in \mathbb{R}$, set of repelled particles $r_i \subseteq P$. The motion of the particle in the space is governed by equation:
\begin{equation}
s_i \left( t \right) = -m_i \left( t \right) A_i \left( t \right) + \mu_i \left( t \right) R_i \left( t \right) + \eta \left( t \right)
\end{equation}
where $A_i(t)$ and $R_i(t)$ are the result of the attractive and repulsive forces, and $\eta(t)$ is a random (small) noise term. The attractive motion term is of the form:
\begin{equation}
A_i(t) = \sum \limits_{p_j \in P - r_i} m_j(t)(s_i - s_j) \delta_{ij}^A
\end{equation}
where
\begin{equation}
\delta_{ij}^A = \begin{cases} 
\Vert s_i - s_j \Vert ^{-2} & \Vert s_i - s_j \Vert ^2 > \theta^A \\
0 & \Vert s_i - s_j \Vert ^2 \leq \theta^A 
\end{cases}
\end{equation}
The repulsive term is of the form
\begin{equation}
R_i(t) = \sum \limits_{p_j \in r_i} \mu_j(t)(s_i - s-j) \delta_{ij}^R
\end{equation}
where
\begin{equation}
\delta_{ij}^R = \begin{cases} 
\Vert s_i - s_j \Vert ^{-2} & \Vert s_i - s_j \Vert ^2 \leq \theta^R \\
0 & \Vert s_i - s_j \Vert ^2 > \theta^R 
\end{cases}
\end{equation}
The two delta functions have different roles in the definition of the GS. The attractive $\delta^A$ corresponds to the inverse of the distance and is the strength of attraction. To avoid singular values when two particles are close to zero distance we set a threshold $\theta^A$ which determines the region around the particles where the motion due to attraction forces disappear. The repulsive $\delta^R$ defines for each $ij$, the maximum extension of the repulsive forces, which are short range forces. The threshold $\theta^R$ determines the region around the particles where the repulsive forces are active.

A vertex particle of a GS-GC reaches zero velocity if and only if it is at distance below $\theta^A$ of a color particle and no repulsive particle is in $\theta^R$ range.

A global state of the GS-GC is stationary if and only if all vertex particles are placed in the neighbourhood of some color particle without any repulsive particles located at the same color particle neighbourhood.
If the graph's chromatic number $M^\star$ is smaller than or equal to the number of color particles $M^\star \leq M$, there will be a non-empty set of stationary states of the GS-GC.

Any stationary state of the GS-GC corresponds to a graph colouring.
If the graph's chromatic number is greater than the number of color particles, there are no stationary states in the GS-GC.

These conditions mean that it is always possible to (given enough time) find the chromatic number for a graph. The algorithm to do this is outlined in the following sections.

\section{Genetic Algorithm}
\subsection{Introduction}
The genetic algorithm presented in this review is taken from this paper \cite{bib:GeneticAlg}. The paper presents a hybrid technique that applies a genetic algorithm followed by wisdom of crowds voting. The algorithm uses a variety of parent selection, child production and mutation steps. The methods of parent selection, child production and mutation vary according to the state of the fitness of the overall system. This results in an algorithm that is resistant to capture by local optima and allows for the solution space to 'jitter' around before falling upon a global solution. 

\subsection{Algorithm Discussion}
The algorithm is presented in further detail in the Implementations section. The following discussion focuses on algorithm design and fine tuning for the GCP. The algorithm makes careful note that local optima are a concern to this algorithm, and the danger posed to any GCP solving heuristic algorithm is falling into a local optima and not being able to recover. The algorithm uses several approaches to avoid local optima and to that end, a number of different factors considered. 

The crossover function, the method that produces a new chromosome from two previously existing chromosomes is the result of a tournament between parents, where 4 chromosomes are chosen at random and the best for the first pair becomes the first parent, and the best of the second pair becomes the second parent. The crossover function takes the chromosomes of the parents and cuts them both at the same point. The new chromosome is the first part of the first parent's chromosome with the second part of the second parents chromosome appended to the end. This results in a new chromosome of the same length as the original parents chromosomes, and a potentially vastly different 'bad edge' count. This new chromosome can then be subjected to mutation (at some rate, given to be 0.7 here). This mutation method examines each vertex in the chromosome and tries to improve it if it is part of a 'bad edge'. 

This process is shown to be very efficient at improving the solution space. It falls prone to the problem of local optima though, so an alternative parent selection and mutation method are employed. These new methods simply take the best solution and mutate it. The mutation allows the solution to become worse, instead of simply trying to improve the solution, each bad edge is allowed to become any colour. This results in the best solution having the possibility of improving, but most likely actually becoming worse. This is not a bad thing though, as the solution space is shown to improve over time. 

If no valid solution to the colouring is found after a set number of generations (given to be 20,000 in this paper), then a wisdom of crowds approach is implemented, where the best portion (given to be half) of the chromosomes vote to create a new chromosome. This aggregated chromosome is checked to see if it contains a solution to the graph, if not, it becomes the first chromosome is the next round of generations, and all other chromosomes are discarded. 

\subsection{Observations}
This algorithm can become very stagnated when the best solution is very close to an optimal solution. To improve this part of the algorithm, the algorithm implementation was changed to instead choose the best parent and a random chromosome on some chance. 
\section{Flower Pollination}

%Refs

% @incollection{yang2012flower,
%   title={Flower pollination algorithm for global optimization},
%   author={Yang, Xin-She},
%   booktitle={Unconventional computation and natural computation},
%   pages={240--249},
%   year={2012},
%   publisher={Springer}
% }
% @INPROCEEDINGS{7175923, 
% author={Bensouyad, M. and Saidouni, D.}, 
% booktitle={Cybernetics (CYBCONF), 2015 IEEE 2nd International Conference on}, 
% title={A discrete flower pollination algorithm for graph coloring problem}, 
% year={2015}, 
% pages={151-155}, 
% keywords={graph colouring;optimisation;continuous optimization;discrete flower pollination algorithm;discrete optimization;flower pollination process;graph coloring problem;nature inspired algorithm;Color;Complexity theory;Law;Optimization;Sociology;Statistics;Discrete Optimization;Flower Pollination Algorithm;Graph coloring}, 
% doi={10.1109/CYBConf.2015.7175923}, 
% month={June},}

The flower pollination algorithm\cite{yang2012flower} (FPA) was first described by Xin-She Yang\footnote{Department of Engineering, University of Cambridge}. It is a metaheuristic global optimisation algorithm inspired by the pollination process of flowering plants. Although originally formulated to solve continuous optimisation problems, Meriem Bensouyad\ref{1} and DjamelEddine Saidouni\ref{1} propose a discrete version of FPA for solving the graph coloring problem\cite{7175923}.\\
\footnote{MISC Laboratory Constantine 2 University}\label{1}
%do the intro?
Pollination can take two major forms: abiotic and biotic. About 90\% of flowers belong to the class of biotic pollinating flowers, that is, their pollen is transferred by pollinators such as insects, birds, bats and other animals. The remaining 10\% of flowers use abiotic pollination, which does not require any pollinators. Wind and diffusion in water help pollination of such flowering plants, of which grass is a good example. Pollinators, or sometimes called pollen vectors, can be very diverse. It is estimated that there are at least 200,000 varieties of pollinators.\\
A concept that comes up when talking about pollinators is \emph{constancy}, which refers to the tendency of some pollinators to visit certain flower species while bypassing others. It has been conjectured that constancy has evolutionary advantages for both the flower species and the pollinator species.\\
Pollination can be achieved by self-pollination or cross-pollination. Cross-pollination, or allogamy, means pollination can occur from pollen of a flower of a different plant, while self-pollination is the fertilisation of one flower, such as peach flowers, from pollen of the same flower or different flowers of the same plant, which often occurs when there is no reliable pollinator available. %this need to be rewritten, got these papers suck
\\
Biotic, cross-pollination may occur at long distances, and the pollinators such as bees, bats, birds and flies can fly a long distance, thus they can be considered as the global pollination. In addition, bees and birds may behave as L\~evy flight behaviour, with jump or fly distance steps obey a L\~evy distribution. Furthermore, flower constancy can be used an incremental step using a similarity or difference of two flowers.\\~\\
JESUE FUCK THESE PAPERS ARE BADLY WRITTEN\\~\\

\subsection{Flower Pollination Algorithm}

The authors of the second paper do not deviate at this stage from the original, and both introduce the actual algorithm by arguing that the above characteristics can be idealised as follow:
\begin{enumerate}[1.]
\item Biotic and cross-pollination is considered as global pollination process with pollen-carrying pollinators performing L\~evy flights.
\item Abiotic and self-pollination are considered to be local pollination.
\item Flower constancy can be considered as the reproduction probability is proportional to the similarity of two flowers involved.
\item Local pollination and global pollination is controlled by a switch probability $p \in [0,1]$. Due to the physical proximity and other factors, such as wind, local pollination can have a significant fraction $p$ in the overall pollination activities.
\end{enumerate}
For simplicity's sake, the authors assume that each plant only has one flower and each flower only produces one pollen gamete\footnote{sperm cells}. In reality the number of flowers and pollen gametes per can vary widely between species and individual plants. This means that a solution, which we denote $x_i$, is equivalent to a flower and/or a pollen gamete\footnote{There is no distinction between a flower and the pollen it produces, all that matters is how the pollen is used.}. The original author believes that extending the algorithm to multiple pollen gametes and multiple flowers (for multiobjective optimisation problems - I did not know that was a thing) should be easy.\\
The FPA is described by breaking it down into 2 key steps: global pollination and local pollination.\\
For global pollination it is argued that the possibility of large travel distances for pollens and flower constancy give rise to the following step rule:
\begin{equation}
x_i^{t+1} = x_i^{t} + L(x_i^{t} - g_*)
\end{equation}
where $x_i^{t}$ is the pollen $i$ or solution vector $x_i$ at iteration $t$ and $g_*$ is the current best solution found amongst all solutions at the current iteration. \\
\begin{quotation}
At this point we have to mention that the argument for global pollination being based on the current best solution, ``the fittest flower'', is non-existent - it just appears in the middle of talking about travel distances and constancy without any justification.
\end{quotation}
The parameter $L$ is supposedly the ``strength'' of the pollination, which is a step size drawn from a L\~evy distribution
\begin{equation}
L ~ \frac{\lambda \gamma \left( \lambda \right) \sin \left( \pi \lambda / 2 \right) }{\pi} \frac{1}{s^{1+\lambda}}, \, (s \gg s_0 > 0)
\end{equation}
Here $\gamma(\lambda)$ is the standard gamma function, and this distribution is valid for large steps $s>0$. Yang reports using $\lambda = 1.5$ for the simulations in the original paper, the second paper's authors (for GCP) do not report what value they used\footnote{we will talk about why this is unhelpful later}.\\
The local pollination, which also includes an allowance for constancy apparently, can be represented as
\begin{equation}
x_i^{t+1} = x_i^{t} + \epsilon(x_j^{t} - x_k^{t}),
\end{equation}
where $x_j^{t}$ and $x_k^{t}$ are pollens from the different flowers of the same plant species and $\epsilon$ is drawn from a uniform distribution in $[0,1]$. (WE MAY HAVE FUCKED THIS UP? WTF IS THE SAME PLANT SPECIES IN THIS ALGORITHM?)\\
Most flower pollination activities can occur at both local and global scale. In practice, adjacent flower patches or flowers in the not-so-far-away neighbourhood are more likely to be pollinated by local flower pollens than those far away. For this, we use a switch probability $p$ to switch between common global pollination and intensive local pollination. (I do not know what ``common'' and ``intensive'' mean, they are never explained.) In the original paper (Yang) the author describes using an initial $p - 0.5$ and then performing a parametric study which found that $p = 0.8$ worked best for most applications. Again, the authors of the second paper did not discuss actual values.\\






